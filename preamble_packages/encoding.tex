%%==============================================================================
%% Encoding
%%==============================================================================
\usepackage[
% ascii,
% latin1,
% latin2,
% latin3,
% latin4,
% latin5,
% latin9,
% latin10,
% decmulti,%% DEC Multinational Character Set encoding
% cp850,
% cp852,
% cp858,
% cp437,
% cp437de,
% cp865,
% applemac,%% Macintosh encoding
% macce,%% Macintosh Central European code page
% next,%% Next encoding
% cp1250,%% Windows 1250 (central and eastern Europe)
% cp1252,%% Windows 1252 (Western Europe)
% cp1257,%% Windows 1257 (Baltic)
% ansinew,%% Windows 3.1 ANSI endocing
  utf8,%% Unicode UTF-8 encoding
  ]{inputenc} %% alternativ: \inputencoding{utf8}

%\usepackage[T1]{fontenc}%%Zeichenbelegung der verwendeten Ausgabeschrift
