%% === The LaTeX standard tools bundle ===

%% Place text after the current page.
%\usepackage{afterpage}

%% Extended versions of the environments array, tabular and tabular*.
\usepackage{array}

%% Access bold math symbols.
%\usepackage{bm}

%% Infix arithmetic expresions.
%\usepackage{calc}

%% Alignment on \emph{decimal points} in tabular entries. Requires array.
%\usepackage{dcolumn}

%% Adds large delimiters around arrays. Requires array.
%\usepackage{delarray}

%% Extended version of the enumerate environment.
%\usepackage{enumerate}

%% may be used to control TeX's missing file error loop.
%\usepackage{fileerr}

%% Package and test file for producing \emph{font samples}
%\usepackage{fontsmpl}

%% Place footnotes in the right hand column in two-column mode.
%\usepackage{ftnright}

%% Finer control over horizontal rules in tables. Requires array.
%\usepackage{hhline}

%% Indent the first paragraph of sections etc.
%\usepackage{indentfirst}

%% Produces an overview of the layout of the current document.
%\usepackage{layout}

%% Multipage tables.(Does not require array, but uses extended features)
\usepackage{longtable}

%% Typeset text in columns, with the length of the final columns balanced.
%\usepackage{multicol}

%% Not recomended for new packages, but may help when updating old files.
%\usepackage{rawfonts}

%% Unified interface to shell escape
%\usepackage{shellesc}

%% Draft mode showing the keys used by \cs{label}, \cs{ref}, \cs{cite} etc.
%\usepackage{showkeys}

%% Selective handling of package options. (Used in rawfonts.sty.)
%\usepackage{somedefs}

%% Defines tabularx environment (similar to tabular*)
\usepackage{tabularx}

%% Flexible definition of theorem-like environments.
%\usepackage{theorem}

%% helps to suppress and to control the amount of tracing output
%\usepackage{trace}

%% Smart handling of page references.
%\usepackage{varioref}

%% Flexible version of verbatim environment.
\usepackage{verbatim}

%% eXternall References.
%\usepackage{xr}

%% avoids the common mistake of missing spaces after command names.
%\usepackage{xspace}
