%%==============================================================================
%% Glossaries
%%==============================================================================
%%    1. Add the xindy option to the glossaries package option list.
%%    2. Add \makeglossaries to your preamble
%%    3. Put \printglossary[⟨options⟩] where you want your glossary to appear
%%    4. Run LATEX on your document.
%%    5. Run xindy
%%        xindy -L german -C din5007-utf8 -I xindy -M myDoc\
%%        -t myDoc.glg -o myDoc.gls myDoc.glo
%%    6. Run LATEX on your document again.
%%
%% Defining New Glossaries:
%%  \newglossary[⟨log-ext⟩]{⟨label⟩}{⟨in-ext⟩}{⟨out-ext⟩}{⟨title⟩} [⟨counter⟩]
%%    ⟨label⟩
%%    ⟨in-ext⟩ and ⟨out-ext⟩ specify extensions to input and output files
%%    ⟨title⟩ is the default title
%%    ⟨counter⟩ specifies which counter to use for associated number lists
%%    ⟨log-ext⟩ specifies extension for makeindex or xindy transcript file
%%    examples: \newglossary[slg]{symbols}{sls}{slo}{\glssymbolsgroupname} %
%%          \newglossary[nlg]{numbers}{nls}{nlo}{\glsnumbersgroupname} %
%%          \newglossary[ilg]{index}{ind}{idx}{\indexname} %
%%
%% Defining Glossary Entries:
%%    \newglossaryentry{⟨label⟩}{⟨key=value list⟩}
%%    \longnewglossaryentry{⟨label⟩}{⟨key=value list⟩}{⟨long description⟩}
%%      \newacronym[⟨key-val list⟩]{⟨label⟩}{⟨abbrv⟩}{⟨long⟩}
%%
%% Setting Options After the Package is Loaded:
%%    \setupglossaries{⟨key-val list⟩}
%%
%% Displaying a glossary:
%%    \printglossaries
%%    \printglossary[⟨options⟩]
%%    \printglossary[type=\acronymtype]
%%==============================================================================

\usepackage[
%%---- General Options: --------------------------------------------------------
%  nowarn,%% supresses all warnings
%  noredefwarn,%% supresses redefining warnings
%  sanitizesort=false,%% sanitize sort value (default:true)
%  nohypertypes={},%% suppress entry hyperlinks for particular glossary
%  hyperfirst=false,%% hyperlink on first use (default:true)
  savenumberlist=true,%% store number list for each entry (default=false)
%%---- Sectioning, Headings and TOC Options: -----------------------------------
  toc,%% Add glossaries to TOC
%  numberline,%% align TOC entry with numbered section titles
%  section=section,%% make glossaries appear in the named sectional unit
%  ucmark=true,%% set \glsglossarymark uses \MakeTextUppercase
%  numberedsection=autolable,%% [false|nolable|autolabel]
%%---- Glossary Appearance Options: --------------------------------------------
%  entrycounter=true,%% level 0 glossary entries will be numbered
%  counterwithin={x},%% counter will be reset when x is incremented
%  subentrycounter=true,%% level 1 glossary entry will be numbered
%  style=list,%% glossary style to use (default:list)
%  nolong,%% prevents automatically loading glossary-long
%  nosuper,%% prevents automatically loading glossary-super
%  nolist,%% prevents automatically loading glossary-list
%  notree,%% prevents automatically loading glossary-tree
%  nostyles,%% prevents all predefined styles from being loaded
%  nonumberlist,%% suppress associated numberlists
%  seeautonumberlist,%% will also suppress any cross-referencing
%  counter=page,%% default counter to use in the number lists
  nopostdot=true,%% suppresses default post description dot
%  nogroupskip=true,%% suppresses default vertical gap between groups
%%---- Sorting Options: %-------------------------------------------------------
%  sort=standard,%% [standard|def|use]
%  order=letter,%% [word|letter] (default:word)
  xindy={%
    language=german,
    codepage=utf8,
    },
%  xindygloss,%% equivalent to xindy={}
%  xindynoglsnumbers,%% equivalent to xindy={glsnumbers=false}
%%---- Acronym Options: --------------------------------------------------------
  acronym,%% \newglossary[alg]{acronym}{acr}{acn}{\acronymname}
  shortcuts,%% provides shortcut commands for acronyms
%%---- Other Options: ----------------------------------------------------------
  symbols,%% \newglossary[slg]{symbols}{sls}{slo}{\glssymbolsgroupname}
%  numbers,%% \newglossary[nlg]{numbers}{nls}{nlo}{\glsnumbersgroupname}
%  index,%% \newglossary[ilg]{index}{ind}{idx}{\indexname}
]{glossaries}   %% Create glossaries and lists of acronyms

%\usepackage{glossaries-prefix}

\newignoredglossary{common}

\makeglossaries
