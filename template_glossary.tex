%%==============================================================================
%% Glossary
%%==============================================================================
\selectlanguage{ngerman}

\newglossaryentry{gls_example}{ %
  name={<name>},  %% The name of the entry (as it will appear in the glossary)
  description={ %
    <description> %
    \nopostdesc %% to suppress the description terminator for this entry
  },
  descriptionplural={<descriptionplural>},  %%
% parent={}, %% label of parent entry, must be defined before its sub-entries
  text={<text>}, %%If this field is omitted, the value of the name key is used
  first={<first>}, %% first use with \gls
  plural={<plural>},
  firstplural={<firstplural>},
  symbol={},
  symbolplural={},
% sort={}, %%
% type={},
% nonumberlist=true, %% (true|false)
% see=[see also]{<label>}, %% Cross-reference another entry
% long={},            %% acronym only
% longplural={<longplural>},    %% acronym only
% short={},           %% acronym only
% shortplural={<shortplural>},  %% acronym only
% prefix={<prefix>},              %% package: glossaries-prefix
% prefixplural={<prefixplural>},        %% package: glossaries-prefix
% prefixfirst={<prefixfirst>},        %% package: glossaries-prefix
% prefixfirstplural={<prefixfirstplural>},  %% package: glossaries-prefix
}

\ifoptionfinal{\glsmoveentry{gls_example}{common}}{}
